\documentclass[a4paper, 12pt]{article}
\usepackage[utf8]{inputenc}
\usepackage[english,russian]{babel}
\usepackage[warn]{mathtext}
\usepackage{graphicx}
\usepackage{float}
\usepackage{multirow}
\restylefloat{table}
\usepackage{amsmath}
\usepackage{floatflt}
\usepackage[T2A]{fontenc}
\usepackage[left=20mm, top=20mm, right=20mm, bottom=20mm, footskip=10mm]{geometry}

\tolerance 1414
\hbadness 1414
\emergencystretch 1.5em
\hfuzz 0.3pt        % размер максимального переполнения без warning'a
\widowpenalty=10000 % запрещает одиночную строку абзаца в начале страницы
\vfuzz \hfuzz
\raggedbottom       % если на странице мало содержимого, добавить пустое место в конце, а не в середине страницы



\begin{document}

\begin{titlepage}
	\centering
	\vspace{5cm}
	{\scshape\LARGE московский физико-технический институт (национальный исследовательский университет) \par}
	\vspace{6cm}
	{\scshape\Large Лабораторная работа 4.3.1 \par}
	{\huge\bfseries Дифракция света \par}
	\vspace{1cm}
	\vfill
\begin{flushright}
	{\large Б03-102}\par
	\vspace{0.3cm}
	{\LARGE Куланов Александр}
\end{flushright}
	

	\vfill


	Долгопрудный, 2023 г.
\end{titlepage}

\begin{itemize}
	\item \textbf{Цель работы:} исследовать явления дифракции Френеля и Фраунгофера на одной и двух щелях, изучить влияние дифракции на разрешающую способность оптических инструментов; проверить теоретические соотношения для положения максимумов при дифракции Френеля и Фраунгофера
    \item \textbf{В работе используются:} оптическая скамья, ртутная лампа, светофильтр, щели с регулируемой шириной, рамка с вертикальной нитью, экран с двойной щелью, микроскоп на поперечных салазках с микрометрическим винтом, зрительная труба
\end{itemize}

\section{Экспериментальная установка}
\subsection*{Дифракция Френеля}



\subsection*{Дифракция Фраунгофера на щели}

\subsection*{Дифракция Фраунгофера на двух щелях}

\subsection*{Влияние дифракции на разрешающую способность оптического инструмента}



\end{document}