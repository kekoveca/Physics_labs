\documentclass[a4paper, 12pt]{article}
\usepackage[utf8]{inputenc}
\usepackage[english,russian]{babel}
\usepackage[warn]{mathtext}
\usepackage{graphicx}
\usepackage{float}
\usepackage{multirow}
\restylefloat{table}
\usepackage{amsmath}
\usepackage{floatflt}
\usepackage[T2A]{fontenc}
\usepackage[left=20mm, top=20mm, right=20mm, bottom=20mm, footskip=10mm]{geometry}

\tolerance 1414
\hbadness 1414
\emergencystretch 1.5em
\hfuzz 0.3pt        % размер максимального переполнения без warning'a
\widowpenalty=10000 % запрещает одиночную строку абзаца в начале страницы
\vfuzz \hfuzz
\raggedbottom       % если на странице мало содержимого, добавить пустое место в конце, а не в середине страницы



\begin{document}

\begin{titlepage}
	\centering
	\vspace{5cm}
	{\scshape\LARGE московский физико-технический институт (национальный исследовательский университет) \par}
	\vspace{6cm}
	{\scshape\Large Лабораторная работа 4.7.3 \par}
	{\huge\bfseries Поляризация света \par}
	\vspace{1cm}
	\vfill
\begin{flushright}
	{\large Б03-102}\par
	\vspace{0.3cm}
	{\LARGE Куланов Александр}
\end{flushright}
	

	\vfill


	Долгопрудный, 2023 г.
\end{titlepage}

\begin{itemize}
	\item \textbf{Цель работы:} оптическая скамья с осветителем; зеленый светофильтр; два поляроида; черное зеркало; полированная эбонитовая пластинка; стопа стеклянных пластинок; слюдяные пластинки разной толщины; пластинки в $1/4$ и $1/2$ длины волны; пластинка в одну длины полны для зеленого цвета (пластинка чувствительного оттенка)
    \item \textbf{В работе используются:} ознакомление с методами получения и анализа поляризованного света.
\end{itemize}
\section{Теоретические сведения}

В работе изучаются свойства поляризованного света. В линейно поляризованной световой волне пара векторов $\vec{E}$ и $\vec{H}$ не изменяет с течением времени своей ориентации. Плоскость $\vec{E}$, $\vec{S}$ называется в этом случае плоскостью колебаний. Наиболее общим типом поляризации является \textit{эллиптическая поляризация}. В эллиптически поляризованной световой волне конец вектора $\vec{E}$ (в данной точке пространства) описывает некоторый эллипс.
	
	При теоретическом рассмотрении различных типов поляризации часто бывает удобно проектировать вектор $\vec{E}$ в некоторой точке пространства на два взаимно перпендикулярных направления. В том случае, когда исходная волна была поляризованной, $E_x$ и $E_y$ когерентны между собой и могут быть записаны в виде
	\begin{equation}
	\begin{cases}
	E_x = E_{x_0}\cos(kz - \omega t),\\
	E_y = E_{y_0}\cos(kz - \omega t-\varphi),
	\end{cases}
	\end{equation}
	где амплитуды $E_{x_0}$, $E_{y_0}$, волновой вектор $k$, частота $\omega$ и сдвиг фаз $\varphi$ не зависят от времени. Формулы (1) описывают монохроматический свет. Немонохроматический свет может быть представлен суммой выражений типа (1) с различными значениями частоты $\omega$.
	
	Ориентация эллипса поляризации определяется отношением амплитуд $E_{y_0}/E_{x_0}$ и разностью фаз $\varphi$. В частности, при $\varphi = 0, \pm\pi$ эллипс вырождается в отрезок прямой (линейная поляризация). При $\varphi = \pm\pi/2$ главные оси эллипса совпадают с осями $x$, $y$. Если при этом отношение амплитуд $E_{y_0}/E_{x_0} = 1$, эллипс поляризации вырождается в окружность.	
	
	В плоскости $z = z_0$ вектор $\vec{E}$ волны (1) вращается против часовой стрелки (при наблюдении навстречу волне), если $0 < \varphi < \pi$. В этом случае говорят о левой эллиптической поляризации волны. Если же
	$\pi < \varphi < 2\pi$, вращение вектора $\vec{E}$ происходит по часовой стрелке, и волна имеет правую эллиптическую поляризацию.
	
	
	В фиксированный момент времени $t = t_0$ концы вектора $\vec{E}$ при различных $z$ лежат на винтовой линии. При этом для левой эллиптической поляризации образуется левый винт, а для правой --- правый винт.
	
	\textbf{Методы получения линейно поляризованного света.} Для получения линейно поляризованного света применяются \textit{поляризаторы}. Направление колебаний электрического вектора в волне, прошедшей через поляризатор, называется \textit{разрешенным направлением поляризатора}. Всякий поляризатор может быть использован для исследования поляризованного света, т. е. в качестве анализатора. Интенсивность $I$ линейно поляризованного света после прохождения через анализатор зависит от угла, образованного плоскостью колебаний с разрешенным направлением анализатора:
	\begin{equation}
	I = I_0 \cos^2\alpha.
	\end{equation}
	Соотношение (2) носит название закона Малюса. Опишем способы получения плоскополяризованного света, используемые в работе.
	
	\textbf{Отражение света от диэлектрической пластинки}. Отраженный от диэлектрика свет всегда частично поляризован. Степень поляризации света, отраженного от диэлектрической пластинки в воздух, зависит от показателя преломления диэлектрика $n$ и от угла падения $i$. Как следует из формул Френеля, полная поляризация отраженного света достигается при падении под углом Брюстера, который определяется соотношением
	\begin{equation}
	\text{tg}i = n.
	\end{equation}
	В этом случае плоскость колебаний электрического вектора в отраженном свете перпендикулярна плоскости падения.
	
	\textbf{Преломление света в стеклянной пластинке}. Поскольку отраженный от
	диэлектрической пластинки свет оказывается частично (или даже полностью) поляризованным, проходящий свет также частично поляризуется. Преимущественное направление колебаний электрического вектора
	в прошедшем свете совпадает с плоскостью преломления луча. Максимальная поляризация проходящего света достигается при падении под
	углом Брюстера. Для увеличения степени поляризации преломлённого
	света используют стопу стеклянных пластинок, расположенных под углом Брюстера к падающему свету.
	
	\textbf{Преломление света в двоякопреломляющих кристаллах}. Некоторые кристаллы обладают свойством двойного лучепреломления. Это связано с различием поляризуемости молекул в разных направлениях (диэлектрическая проницаемость $\varepsilon$ определяет показатель преломления среды $n$).
	Двоякопреломляющий кристалл называют одноосным, если в нём существует одно направление с экстремальным значением $\varepsilon$, а в других (перпендикулярных) направлениях значения $\varepsilon$ одинаковы. Направления вдоль осей эллипсоида называют главными, одно из них --- c экстремальным значением $\varepsilon$ --- оптической осью. Преломляясь в таких кристаллах, световой луч разделяется на два луча со взаимно перпендикулярными плоскостями колебаний. Отклоняя	один из лучей в сторону, можно получить плоскополяризованный свет, --- так устроены поляризационные призмы (Николя, Глана).

\section{Выполнение работы}
\subsection{Определение разрешённых направлений поляроидов}
Разместим па оптической скамье осветитель, поляроид 1 и чёрное зеркало. Свет, отражённый от зеркала, рассматриваем сбоку, расположив глаз таким образом, чтобы вблизи оси вращения зеркала можно было увидеть изображение диафрагмы осветителя. Поворачивая поляроид вокруг направления луча, добьёмся наименьшей яркости отражённого пятна. Оставим поляроид в этом положении и вращением зеркала вокруг вертикальной оси снова добьёмся минимальной интенсивности отражённого луча.
Вместо чёрного зеркала поставим второй поляроид. Скрестим их, определим разрешённое направление второго поляроида.

\subsection{Определение показателя преломления эбонита}
Направление разрешенных колебаний первого поляроида горизонтально. Найдем угол брюстера: такой угол, при котором интенсивность отраженного луча минимальна.
\begin{equation}
	n = \tg \varphi = \tan (56^\circ) \approx 1,5
\end{equation}

\subsection{Исследование стопы}
Поставим стопу стеклянных пластинок вместо эбонитового зеркала и подберем для неё такое положение, при котором свет падает на стопу под углом Брюстера. Осветим стопу неполяризованным светом и, рассматривая через поляроиды свет, отражённый от стопы, определим ориентацию вектора $ \mathbf{E} $ в отражённом луче; затем определим характер поляризации света в преломлённом луче. Получаем, что преломленные лучи горизонтальные, отраженные вертикальные

\subsection{Определение главных плоскостей двоякопреломляющих пластин}
Поставим кристаллическую пластинку между скрещенными поляроидами $ P_1 $ и $ P_2 $. Вращая пластинку вокруг направления луча и наблюдая за интенсивностью света, проходящего сквозь второй поляроид, определим, при каком условии главные направления пластинки совпадают с разрешёнными направлениями поляроидов. Повторим опыт для второй пластинки. Минимумы и максимумы интенсивности чередуются через 45°, главные плоскости пластин совпадают с разрешенными направлениями поляроидов при максимальной интенсивности.

Выделим пластины $ \lambda / 2 $, $ \lambda / 4 $. Добавим к предыдущей схеме зелёный фильтр и установим разрешённое направление первого поляроида горизонтально, а главные направления исследуемой пластинки -- под углом 45° к горизонтали. Для одной пластинки после поворота свет становится фиолетовым и тускнеет. Тогда это линейная поляризация, пластинка $ \lambda / 2 $. Для другой пластинки свет становится зелёным и не тускнеет. Это пластинка $ \lambda / 4 $ и круговая поляризация.
\subsection{Определение направления вращения светового вектора в эллиптически поляризованной волне}
Снова поставим зеленый фильтр, а за ним между скрещенными поляроидами -- пластинку $( \lambda / 4 )$.
ля определения направления вращения светового вектора в эллипсе установим между поляроидами дополнительную пластинку $ \lambda / 4 $ с известными направлениями «быстрой» и «медленной» осей, ориентированными по осям эллипса поляризации анализируемого света. В этом случае вектор \textbf{E} на выходе будет таким, как если бы свет прошел две пластинки $ \lambda / 4 $: свет на выходе из второй пластинки будет линейно поляризован. Если пластинки поодиночке дают эллипсы, вращающиеся в разные стороны, то поставленные друг за другом, они скомпенсируют разность фаз, и вектор \textbf{E} на выходе останется в первом и третьем квадрантах. Если же световой вектор перешёл в смежные квадранты, значит, эллипсы вращаются в одну сторону.

\section{Выводы}

В этой работе мы узнали про методы получения и анализа поляризованного света. Познакомились с такими приборами, как черное зеркало, различные пластинки и поляроиды. Нашли угол Брюстера для эбонита.

\end{document}