\documentclass[a4paper, 12pt]{article}
\usepackage[utf8]{inputenc}
\usepackage[english,russian]{babel}
\usepackage[warn]{mathtext}
\usepackage{graphicx}
\usepackage{float}
\restylefloat{table}
\usepackage{amsmath}
\usepackage{floatflt}
\usepackage[T2A]{fontenc}
\usepackage[left=20mm, top=20mm, right=20mm, bottom=20mm, footskip=10mm]{geometry}

\tolerance 1414
\hbadness 1414
\emergencystretch 1.5em
\hfuzz 0.3pt        % размер максимального переполнения без warning'a
\widowpenalty=10000 % запрещает одиночную строку абзаца в начале страницы
\vfuzz \hfuzz
\raggedbottom       % если на странице мало содержимого, добавить пустое место в конце, а не в середине страницы



\begin{document}

\begin{titlepage}
	\centering
	\vspace{5cm}
	{\scshape\LARGE московский физико-технический институт (национальный исследовательский университет) \par}
	\vspace{6cm}
	{\scshape\Large Лабораторная работа 3.3.6 \par}
	{\huge\bfseries Влияние магнитного поля на проводимость полупроводников \par}
	\vspace{1cm}
	\vfill
\begin{flushright}
	{\large Б03-102}\par
	\vspace{0.3cm}
	{\LARGE Куланов Александр}
\end{flushright}
	

	\vfill


	Долгопрудный, 2022 г.
\end{titlepage}

\begin{itemize}
	\item \textbf{Цель работы:} Измерение влияния магнитного поля на полупроводники
    \item \textbf{В работе используются:} Стабилизированный источник постоянного тока и напряжения, электромагнит, цифровой вольтметр, aмперметр, миллиамперметр, реостат, измеритель магнитной индукции III1-10, образцы (InSb) монокристаллического антимонида индия n-типа
    
\end{itemize}

\section{Описание установки}
\begin{figure}[H]
    \centering
    \includegraphics[width=0.7\textwidth]{set}
    \caption{Схема установки}
    \label{fig:set}
\end{figure}

В зазоре электромагнита создаётся постоянное магнитное поле. Ток питания магнита подаётся от источника постоянного напряжения GPR-11H30D, регулируется ручками управления источника $\left(R_{1}\right)$ и измеряется амперметром источника $A_{1}$. Магнитная индукция в зазоре электромагнита определяется при помощи измерителя магнитной индукции Ш1-10 (описание прибора расположено на установке).

Образец в форме кольца (диск Корбйно) или пластинки, смонтированный в специальном держателе, подключается к источнику постоянного напряжения 5 В. При замыкании ключа К сквозь образец течёт ток, величина которого измеряется миллиамперметром $A_{2}$ и регулируется реостатом $R_{2}$ Балластное сопротивление $R_{0}$ ограничивает ток через образец. Измеряемое напряжение подаётся на вход цифрового вольтметра $\mathrm{B} 7-78 / 1$

\section{Теоретические сведения}

\section{Приложение}

\end{document}